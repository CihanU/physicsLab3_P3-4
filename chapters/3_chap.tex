\chapter{Execution}
\section{Calibration of the Force Sensor} \label{sec:Calibration}
First, the measurement setup must be calibrated. For this purpose, three calibration weights are selected from a choice of weights. These are first weighed on the balance provided in the laboratory. They are then hung on the cantilever with the strain gauges in alternating arrangements, i.e. alone and in various combinations. The \textsc{Du Noüy-ring}, which is required for later measurements, is removed in the meantime. Care is taken to ensure that the weights do not swing. For each arrangement, the offset button is first pressed and the data recorded for about 30 seconds and stored in a text file.
%
\section{Resolution and Statistics} \label{sec:Statistics}
The variance of the conversion result is analyzed by repeating the procedure from \cref{sec:Calibration}, but without additional weights. The values are recorded and stored over a period of \SI{100}{s}.
%
\section{Measurement of Surface Tension} \label{sec:Surface_Tension}
To avoid unnecessary errors when measuring the surface tension, the \textsc{Du Noüy-ring} is first washed under running water and rinsed with distilled water. A paper towel is used for drying. The ring is not touched after washing and is carefully hung in the hook on the boom. A glass beaker is filled with distilled water to the point where the ring can be completely immersed. The beaker is placed on a height-adjustable platform just below the ring. The ring is now aligned as parallel as possible to the water surface. To stabilize the alignment of the ring, a piece of thin wire is also wrapped around the holding strings. Now the platform is turned upwards until the ring almost touches the water. The offset button is pressed and the recording of the values is started with \textsc{RealTerm}. Now the platform is slowly moved further up until the ring is completely covered with water. Then the platform is lowered again until the ring is no longer in contact with the water and the lamella is torn. The data is stored. This procedure is repeated until 10 measurements have been recorded. After the measurement with pure water has been completed, the behavior is examined when detergent is dissolved in the water. To do this, the tip of a thin wire is dipped in detergent and then the tip is stirred into the water. Again, 10 series of measurements are recorded according to the same scheme. Then a second time detergent is stirred in with the tip of the wire and again 10 measurements are recorded. Finally, the water is tipped away and the beaker is filled with 2-isopropanol. Again, 10 measurements are recorded according to the same scheme. Finally, the diameter of the ring is measured with a caliper.
