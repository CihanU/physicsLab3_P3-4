\chapter{Evaluation}
%
\section{Calibration of the Force Sensor}
Because of a measurement misfortune for the calibration, the common way for determining the calibration factor is inoperable. Therefore another method is improvised. Originally, the calibration factor $ c $ was supposed to be determined by the linear fit of the ADC result-force-curve, while the weight of the different masses would have been converted into forces:
\begin{align}
c=\frac{\Delta n}{\Delta F}, \qquad [c]=\frac{1}{\SI{}{mN}}
\end{align}
Since this measurement failed, the literature value of the surface tension of distilled water has to be used in order to be able to carry out a backward calculation. For that, \cref{eq:surface tension} is transformed into the force $ F_0 $:
\begin{align*}
F_0=\sigma \cdot 4\pi \cdot R_{Ring}
\end{align*}
With
\begin{align}
F_0=F_{max}=\frac{n_{max}}{c}
\label{eq:force}
\end{align}
it follows for distilled water:
\begin{align}
\frac{n_{max}}{c}&=\sigma_{H_2O} \cdot 4\pi \cdot R_{Ring} \nonumber \\
\Leftrightarrow \qquad c&=\frac{n_{max}}{\sigma_{H_2O} \cdot 4\pi \cdot R_{Ring}}
\label{eq:calibration factor}
\end{align}
with $ \sigma_{H_2O}=\SI{72.8}{\frac{mN}{m}} $ at $ 20 \SIUnitSymbolCelsius $ \cite{Eichler.2016}
\\\\
The ADC result $ n_{max} $ can be read from the du Noüy ring method measurements with distilled water. There are 10 values for $ n_{max} $:
\begin{align*}
n_{max,i}=[80321, 77600, 77085, 80827, 76744, 82423, 79322, 77319, 79543, 77873]
\end{align*}
The mean value and the deviation (as standard deviation) are
\begin{align*}
\bar{n}_{max}&=78906\\
\Delta n_{max}&=1883
\end{align*}
\Cref{eq:calibration factor} gives the calibration factor:
\begin{align*}
\fbox{c=\SI{(2875 \pm 73)}{\frac{1}{mN}}}
\end{align*}
The deviation of the calibration factor is calculated as follows:
\begin{align*}
\Delta c&=\left| \frac{\partial c}{\partial n_{max}} \right| \cdot \Delta n_{max} + \left| \frac{\partial c}{\partial R_{Ring}} \right| \cdot \Delta R_{Ring} \\
&=\frac{1}{\sigma_{H_2O} \cdot 4\pi \cdot R_{Ring}} \cdot \Delta n_{max} + \frac{n_{max}}{\sigma_{H_2O} \cdot 4\pi \cdot R_{Ring}^2} \cdot \Delta R_{Ring} \\
&=\frac{1}{\SI{72.8}{\frac{mN}{m}} \cdot 4\pi \cdot \SI{0.03}{m}} \cdot 1883 + \frac{78906}{\SI{72.8}{\frac{mN}{m}} \cdot 4\pi \cdot (\SI{0.03}{m})^2} \cdot \SI{0.00005}{m} \\
&=\SI{68.61}{\frac{1}{mN}}+\SI{4.79}{\frac{1}{mN}} \\
&=\SI{73.4}{\frac{1}{mN}} \approx \SI{73}{\frac{1}{mN}}
\end{align*}
%
\section{Resolution and Statistics}
For comparison of the raw and filtered data of the A/D converter with regard to statistical variations, both are plotted as a function of the time in a stray diagram.\\
As it can be seen, the raw data has a greater scattering. This is also confirmed by the standard deviations, which are as follows:
\begin{align*}
\bar{n}_{raw}&=-21, \qquad \sigma_{raw}=58\\
\bar{n}_{filtered}&=-23, \qquad \sigma_{filtered}=39
\end{align*}
Based on that, the deviation of the force to be read can be determined as
\begin{align*}
\Delta F&=\left| \frac{\partial F}{\partial \bar{n}_{filtered}} \right| \cdot \sigma_{filtered} + \left| \frac{\partial F}{\partial c} \right| \cdot \Delta c \\
&=\frac{1}{c} \cdot \sigma_{filtered} + \frac{\left|\bar{n}_{filtered}\right|}{c^2} \cdot \Delta c \\
&=\frac{1}{\SI{2875}{\frac{1}{mN}}} \cdot 39 + \frac{23}{(\SI{2875}{\frac{1}{mN}})^2} \cdot \SI{73}{\frac{1}{mN}} \\
&=\SI{0.0135}{mN}+\SI{0.0002}{mN} \\
&=\SI{0.0137}{mN} \approx \SI{0.01}{mN}
\end{align*}
The effective resolution of the A/D converter is given by means of \cref{eq:ENOB}:
\begin{align*}
ENOB&=-23-\log_2(39)=-28
\end{align*}
Further, the histograms show that the raw data is more widely spread than the filtered one. The raw data histogram has a binning value of around 40 and the filtered data one of around 20.
%
\section{Du Noüy Ring Method Measurement}
The data is being sent to the PC during the measurement procedure. When the A/D results are converted into the force by way of \cref{eq:force}, the force-time-diagram is obtained.\\
The maximum forces for distilled water are as follows:
\begin{align*}
F_{max,H_2O,i}=[27.9, 27.0, 26.8, 28.1, 26.7, 28.7, 27.6, 26.9, 27.7, 27.1] \quad \text{(all values in \SI{}{mN})}
\end{align*}
The forces are around the mean value
\begin{align*}
\bar{F}_{max,H_2O}=\SI{27.5}{mN}
\end{align*}
,so the calculated value in \cref{eq:calculated force} is approximated well.
